\section{Mereni}
\subsection{Postup měření}%mozna to dat hned pod ten hlavni nadpis
\subsection{Výsledky měření} %TODO co jsem dát aniž by to nebyla tabulka na 3 strany
\subsection{Zpracovani dat}
%TODO magic kterej je popsanej v vedeckejch studiich ktere jsem cetl



\begin{table}[h!]
	\begin{center}
		\caption{table from .csv file.}
		\label{table1}
		\pgfplotstabletypeset[
		multicolumn names, % allows to have multicolumn names
		col sep=comma, % the seperator in our .csv file
		display columns/0/.style={
			column name=$energie$, % name of first column
			column type={S},string type},  % use siunitx for formatting
		display columns/1/.style={
			column type={S},string type},
		display columns/2/.style={
			column type={S},string type},
		display columns/3/.style={
			column type={S},string type},
		display columns/4/.style={
			column type={S},string type},
		display columns/5/.style={
			column type={S},string type},
		every head row/.style={
			before row={\toprule}, % have a rule at top
			after row={
				\si{\kilo\electronvolt}\\ % the units seperated by &
				\midrule} % rule under units
		},
		every last row/.style={after row=\bottomrule}, % rule at bottom
		]{rocnikovkadata.csv} % filename/path to file
	\end{center}
\end{table}