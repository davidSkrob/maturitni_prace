
\section*{Seznam použitých značek, symbolů a zkratek}
\begin{spacing}{1.2}
	\begin{list}{}{}
		\item $A_0$ -- aktivita kalibračního zářiče k referenčnímu datu
		\item $c$ -- rychlost světla ve vakuu %-- \SI{299792458}{\meter\per\second}
		\item \e -- elektron%, $m_e = \SI{9.1093837e-31}{\kilogram}$%https://physics.nist.gov/cgi-bin/cuu/Value?me
		\item $e^+$ -- pozitron, antičástice k \e	
		\item $\varepsilon$ -- efektivita
		\item \SI{}{\electronvolt} -- elektron volt -- je to energie, kterou má jeden elektron urychlený napětím $\SI{1}{\volt}$
		\item FWHM -- Full Width at Half Maximum -- šířka na poloviční výšce peaku
		\item $\gamma$ -- gama záření -- elektromagnetické záření, původem z jaderných reakcí
		\item $h$ -- Planckova konstanta -- \SI{6.62607015e-34}{\joule\second} 
		\item $\hbar$ -- redukovaná Plankova konstanta -- $\hbar = \frac{h}{2\pi}$
		\item HPGe -- High-Purity Germanium -- detektor z velmi čistého germania
		\item $I_\gamma$ -- intenzita gama přechodu
		\item $\lambda$ -- rozpadová konstanta %-- $\frac{\ln(2)}{T_{1/2}}$
		\item $n$ -- registrovaná četnost
		\item $N$ -- skutečná četnost
		\item $\nu$ -- frekvence fotonu
		\item $S_{peaku}$ -- plocha peaku, bez pozadí
		\item $t_0$ -- doba mezi referenčním datem a datem měření
		\item $T_{1/2}$ -- poločas přeměny -- doba, za kterou se přemění $\frac{1}{2}$ celkového počtu jader
		\item $t_{live}$ -- čistý čas měření
		\item $t_{real}$ -- celková doba měření
		\item $\tau$ -- mrtvá doba detektoru	
	\end{list}
\end{spacing}
\newpage
\addcontentsline{toc}{section}{Seznam použitých odborných výrazů}
\section*{Seznam použitých odborných výrazů}
\begin{list}{}{}
	\item [Fotoelektrický jev] -- jev, při kterém foton vytrhne elektron z elektronového obalu. Popsal jej Albert Einstein v roce 1905. Nezáleží na intenzitě světla, pouze na jeho frekvenci. 
	\item [Comptonův jev] -- $\gamma$ (popř. rentgenový) foton narazí na \e, předá část své hybnosti \e. Foton (protože ztratí část energie) má v důsledku nižší frekvenci (= větší vlnovou délku), a je vychýlen od původního směru. Objevil jej  Arthur Holly Compton v roce 1923.
	\item [Vytváření páru pozitron elektron] -- Foton s energií alespoň \SI{1020}{\kilo\electronvolt} se v blízkosti atomového jádra přemění na pár $e^{+}$, $e^{-}$. Objevili jej Blackett a Occhialini v roce 1933.
	\item [Elektromagnetické záření] -- postupné vlnění magnetického a elektrického pole. Objevil je Michael Faraday v roce 1845. 
	\item [Spektroskopie] -- obor fyziky, který se snaží zachytit vliv elektromagnetického záření na danou látku.
\end{list}
\newpage

\begin{table}[h!]
	\begin{center}
		\caption{table from .csv file.}
		\label{table1}
		\pgfplotstabletypeset[
		multicolumn names, % allows to have multicolumn names
		col sep=comma, % the seperator in our .csv file
		display columns/0/.style={
			column name=$energie$, % name of first column
			column type={S},string type},  % use siunitx for formatting
		display columns/1/.style={
			column type={S},string type},
		display columns/2/.style={
			column type={S},string type},
		display columns/3/.style={
			column type={S},string type},
		display columns/4/.style={
			column type={S},string type},
		display columns/5/.style={
			column type={S},string type},
		every head row/.style={
			before row={\toprule}, % have a rule at top
			after row={
				\si{\kilo\electronvolt}\\ % the units seperated by &
				\midrule} % rule under units
		},
		every last row/.style={after row=\bottomrule}, % rule at bottom
		]{rocnikovkadata.csv} % filename/path to file
	\end{center}
\end{table}